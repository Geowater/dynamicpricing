%!TEX root = main.tex
\section{Features}
\label{sec:features}

	We explained in the section \ref{sec:models} that the models are based on features. The features define which data could be relevant for the predictions. Therefore, is is important to consider which features will improve our models.
\subsection{Chosen features}
	\subsubsection{Price rank}
	\label{sec:pricerank}
		~\\
		The price rank is one of our basic and most important features, especially because it can be used always, even if we don't have specific information/a specific model for the current product to sell. It represents one of the customers main causes to buy or not to buy an offer. However, this feature works much better in combination with quality, shipping time etc..
	\subsubsection{Amount of offers}
		~\\
		Depending on the size of the competition, higher prices may be charged or lower prices may have to be accepted in order to sell an offer with a high probability. This circumstance should be respected by the model in order to use this as an advantage.
	\subsubsection{Relative price difference to cheapest price (in percent)}
		~\\
		The price difference to the cheapest price can be a good clue for what a good price is. If the difference is little, the customer might decide for a slightly more expensive offer if other conditions such as quality or shipping time are better. The higher the difference is, the higher is the probability that the customer accepts other disadvantages in order to get the best price.
	\subsubsection{Relative price difference to 2nd \& 3rd cheapest price (in percent)}
	\label{sec:pricediff3}
		~\\
		In general, the use of this feature is similar to the previous. However, it is particularly useful if only a single or two competitors have abnormal prices and would adulterate the prediction results. In those cases, the distance cheapest (and maybe second cheapest) is very high, but the distances to other competitors are correspondingly lower.
	\subsubsection{Price}
		~\\
		The following features are not universal anymore but product specific features. Those features we only used in our models we have created for each product.

		The first one is the price. While being on a specific price rank for example not necessarily leads to a concrete price (meaning the \lq{best}\rq\ price), the concrete selling prices do. This is especially helpful, if the different price ranks are wide apart. In those cases, it is not automatically said, if the price should be closer to the next cheaper price or to the next price being more expensive.
	\subsubsection{Quality}
		~\\
		Similar to the price feature, we also introduced the concrete quality as a feature. The influence of this property is of course not as high as that of the price (the difference between the qualities is in most cases relatively small). But even here it is possible, that the best quality and the worst quality in the market situation differ by more than one, so the best price can be influenced by the fact whether our quality is closer to the better one or closer to the worse one.
	\subsubsection{Shipping time}
		~\\
		This feature is pretty similar to the two previous ones. A customer in Europe might be willing to spend more money for a shorter shipping time. If, for example, two competitors have different shipping times, e. g. one from Europe with 1 day and one from Asia with 21 days, rank two can be 2 days (could justify a higher price) but also 20 days. The concrete shipping time is, from this point of view, important for our model to respect.
	\subsubsection{Average price}
		~\\
		This feature describes the average price of the current market situation, without our own potential offer. It can be used to observe the difference of our own selling prices to the corresponding average prices.
	\subsubsection{Average price including our own offer}
		~\\
		This feature is similar to the previous one, but the average price is calculated from all offers, including our potential offer.
	\subsubsection{Average sale prices}
		~\\
		Additionally to the average prices of the current market situation, we also use the average selling price, we have already sold this product for.
	\subsubsection{Absolute price differences to cheapest, 2nd cheapest and 3rd cheapest price}
		~\\
		Because most customers compare and chose from the cheapest offers, we also use the price difference to the three cheapest prices in the current market situation as a feature. A possible result could be, that the best price is e. g. to be scarcely more expensive than the 2nd cheapest price.
\subsection{Non-selected features}
	Some of the features we actually planned to integrate in our merchant weren't suitable for us. The biggest problem is the high failure rate. The probability that e. g. the quality rank or shipping time rank have always the same value is relatively high. If so, we often experienced errors in the learning process leading to program crashes. However they will be briefly explained in the following, because they could be helpful under different conditions:
	\subsubsection{Quality rank}
		~\\
		Depending on the quality rank, customers might accept higher prices. This circumstance is important for us since we do not want the prices too cheap, or, in other words, the price should be as high as the customer still accepts and as low as the customer will decide to buy our offer.
	\subsubsection{Shipping time rank}
		~\\
		Similar to quality rank, also the shipping time can be a reason for the customer to accept higher prices.
	\subsubsection{Prime}
		~\\
		The fact if prime shipping will be provided or not can justify higher prices.
	\subsubsection{Prime Shipping Time}
		~\\
		Even more interesting than the question wheter prime will be provided or not is the question, how long the concrete shipping time would be. Customers could need products particularly fast and might be willing to spend more money for this amenity.
\subsection{Evaluation}
	During our evaluation, we did not identify differences as high as we've has expected. However all features we finally selected brought added value.

	One of the most important features is the price itself (at least for the product-specific models). While different ranks or differences also have their value, the concrete price learned from previous sales led to the highest profit increase. In addition, the price rank brought good profit increase, at least for artificial customers, because buying the product on a specific price rank is an often used approach.

	For customers who attach more importance to other attributes such as quality or shipping time, the corresponding features brought also a high progit increase. Using the combination of all mentioned features gave us a good trade-off for different customer behaviours.
