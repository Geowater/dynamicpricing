%!TEX root = main.tex
\section{Introduction}
More than 40~\% of the German population prefers to buy products online rather than in a local store. \cite{shoppingStatistic} Hence, the number of sellers participating in e-commerce grows drastically. Every dealer tries to maximize his profit. That leads to a large competition amongst the vendors with incessantly changing prices. A single person is longer able to keep track of the market's deveplopment. In order to participate in that battle and maximize the own profit, a dynamic pricing solution has to be applied.

We developed a pricing algorithm, which analyzes the current market situation in a specific interval. The software processes the market data and extracts the features. We evaluated multiple machine learning methods for calculating the optimal price for a product. The most precise technique is the random forest regressor. We use an existing simulation platform\footnote{\href{https://github.com/hpi-epic/masterproject-pricewars}{https://github.com/hpi-epic/masterproject-pricewars}} to evaluate the performance against other vendors. At this platform multiple merchant bots can fight against each other. One or more customers can be set up for simulating different buying behaviors (see section \ref{sec:platform}).

The main problem of offering a product is setting the correct price. On the one hand, many customers might buy it, but the expected profit might be low as well. The difficulty is to find the right balance. Our machine learning approach calculates the probability for selling an item at a certain time. Since we want to maximize the profit, the possible price is entered into the formula.

Additionally, we have implemented a way to avoid extreme price races. If each merchant wants to be the cheapest, products might be sold at dumping prices. That leads to minimal profit or even loss of money. In another scenario like a duopol, vendors might increase their prices tremendously, if the demand and supply are low. Our merchant software determines a price range based on purchasing price to avoid extremely high or extremely low prices.

Online markets are dynamic: They change every time. New products get released, existing ones get hyped or disappear completely. Reacting to changes is one of the biggest challenges of dynamic pricing. Our solution uses weighted history data in order to train the model. Market situations in recent times get more importance than the ones occurred long ago. We use the complete history, because customer behavior won't change drastically over time. Furthermore, we use this information to create a universal model, which predicts prices for new products.

In the following chapter, we present the machine learning approaches we used evaluated. After explaining our features in chapter \ref{sec:features} we used to train our models, we describe some special functionalities of our pricing bot. The performance comparison with other pricing strategies follows in section \ref{sec:performance}. The implemented strategy still has room for improvement, which we discuss in part \ref{sec:future_work}
