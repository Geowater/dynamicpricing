\documentclass[sigconf]{acmart}

\pagestyle{plain} % removes running headers

\usepackage{booktabs}
\usepackage{todonotes}

\usepackage[ruled]{algorithm2e}
\renewcommand{\algorithmcfname}{ALGORITHM}
\SetAlFnt{\small}
\SetAlCapFnt{\small}
\SetAlCapNameFnt{\small}
\SetAlCapHSkip{0pt}
\IncMargin{-\parindent}

\settopmatter{printacmref=false}
\renewcommand\footnotetextcopyrightpermission[1]{}

%\usepackage[ngerman]{babel}
\usepackage[utf8]{inputenc}

\usepackage{graphicx}
\usepackage[T1]{fontenc}


\begin{document}

\title{Data-Driven Demand Learning and Dynamic Pricing Strategies in Competitive Markets} 

\author{Niklas Hoffmann}
\author{Marcel Jankrift} 
\author{Toni Stachewicz}
\affiliation{
  \institution{Hasso-Plattner-Institute}
  \city{Potsdam}
  \country{Germany}
}

\keywords{Dynamic Pricing, Pricing Strategy, Machine Learning, Logistic Regression}

\maketitle

\section{Abstrakt} \textcolor{red}{für Außenstehende, in wenigen Sätzen worum es geht (vor allem wie ihr euch von anderen Gruppen absetzt; Tipps für einen guten Abstrakt: https://plg.uwaterloo.ca/~migod/research/beckOOPSLA.html)}

\section{Introduktion} \textcolor{red}{sollte mit Abstrakt pi*Daumen die ersten Seite füllen; leitet das Thema ein und gibt einen guten Einstieg in die Thematik; motiviert Eure Problemlösung; visiert bitte einen außenstehenden Prof. wie ca. Prof. Naumann an (sprich: Machine Learning muss nicht erklärt werden, aber Dynamic Pricing wie im Seminar wird unbekannt sein)}

\section{Models}
\subsection{Considered models}
\subsubsection{Logistic Regression}
\subsubsection{MLP}
\subsubsection{Random Forest}
\subsubsection{Least Squares}
\subsection{Evaluation}

\section{Features}
\subsection{Chosen features}
\subsubsection{Not chosen features}
\subsection{Evaluation}

\section{Additional functionalities of our merchant}
\subsection{Weighting}
\subsection{Universal model}

\section{Future work}

\section{Conclusion}



\bibliographystyle{ACM-Reference-Format}
\bibliography{bibliography}

\end{document}
